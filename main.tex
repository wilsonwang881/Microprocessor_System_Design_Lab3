\documentclass[11pt,letterpaper,titlepage]{article}

\usepackage{geometry}
\geometry{left=2cm,right=2cm,top=2cm,bottom=3cm}

\usepackage{setspace}
\onehalfspacing

\usepackage{multicol}
\setlength{\columnsep}{3em}

\usepackage{booktabs}

\usepackage[table,x11names]{xcolor}

\usepackage{multirow}

\usepackage{pgfgantt}

\usepackage{listings}

\usepackage{xcolor}
\definecolor{vgreen}{RGB}{104,180,104}
\definecolor{vblue}{RGB}{49,49,255}
\definecolor{vorange}{RGB}{255,143,102}

\lstdefinestyle{C-style}
{
    language=C,
    basicstyle=\small\ttfamily,
    keywordstyle=\color{vblue},
    identifierstyle=\color{black},
    commentstyle=\color{vgreen},
    % numbers=left,
    numberstyle=\tiny\color{black},
    numbersep=11pt,
    tabsize=4,
    moredelim=*[s][\colorIndex]{[}{]},
    literate=*{:}{:}1
}

\lstdefinestyle{txt-style}
{
    basicstyle=\small\ttfamily,
    % numbers=left,
    numbersep=11pt,
    tabsize=4,
    moredelim=*[s][\colorIndex]{[}{]},
    literate=*{:}{:}1
}

\usepackage{tikz}
\usetikzlibrary{shapes.geometric, arrows, positioning, fit,calc}
\newcommand*\circled[1]{\tikz[baseline=(char.base)]{
            \node[shape=circle,draw,inner sep=1pt] (char) {#1};}}
            
\usepackage{hyperref}
\hypersetup{
    colorlinks,
    citecolor=black,
    filecolor=black,
    linkcolor=black,
    urlcolor=black
}

\usepackage{pifont}

\usepackage[toc,page]{appendix}

\pagestyle{empty}
\usepackage{tikz}
\usetikzlibrary{shapes.geometric, arrows}

\usetikzlibrary{mindmap,trees}
\usepackage{verbatim}

\usepackage{indentfirst}
\setlength{\parindent}{2em}

\usepackage{listings}

% \usepackage{hyperref}
\usepackage{chngcntr}
\counterwithin{section}{part}
\renewcommand\thesection{\arabic{section}}

\usepackage{graphicx}

\usepackage{subcaption}

\usepackage{fancyhdr}

\pagestyle{fancy}
\lhead{}
\rhead{}
\lfoot{ECEN 749 Section 601 Lab 3}
\cfoot{\thepage}
\rfoot{@Lei Wang (Wilson)}
\renewcommand{\headrulewidth}{0pt}
\renewcommand{\headwidth}{\textwidth}
\renewcommand{\footrulewidth}{0.4pt}
\newcommand{\RomanNumeralCaps}[1]
    {\MakeUppercase{\romannumeral #1}}

\makeatletter
\newcommand*\@lbracket{[}
\newcommand*\@rbracket{]}
\newcommand*\@colon{:}
\newcommand*\colorIndex{%
    \edef\@temp{\the\lst@token}%
    \ifx\@temp\@lbracket \color{black}%
    \else\ifx\@temp\@rbracket \color{black}%
    \else\ifx\@temp\@colon \color{black}%
    \else \color{vorange}%
    \fi\fi\fi
}
\makeatother

\usepackage{trace}

\begin{document}

\begin{titlepage}
  \centering
	{\scshape\large Texas A\&M University \par}
	\vspace{1cm}
	{\scshape\Large Department of Electrical and Computer Engineering \par}
	\vspace{4cm}
    \vspace{0.5cm}
	{\huge\bfseries ECEN 749 Microprocessor System Design\par}
	\vspace{4cm}
	{\Large Lab 2 Report (Section 601)\par}
	\vspace{1cm}
	{\Large Student: Lei Wang (Wilson)\par}
	\vspace{1cm}
	{\Large UIN: 829009485\par}
	\vspace{1cm}
	{\Large Instructor: Dr. Paul V. Gratz\par}
	\vspace{4cm}
	\vfill

  % Bottom of the page
	{\large Submitted: February 11th, 2020 \par}

\end{titlepage}

\newpage

\tableofcontents{}

\newpage

\part{Introduction}

The lab session aims at teaching students build their own IP, integrate the IP and use C to use the IP. The lab focuses on a simple hardware multiplier implementation.

\part{Procedure}

\part{Results}

\newpage

\part{Conclusion}

\textbf{Q: What is the purpose of the tmp\_reg from the Verilog code provided in lab, and what happens if
this register is removed from the code?}

A:

\textbf{Q: What values of ‘slv\_reg0’ and ‘slv\_reg1’ would produce incorrect results from the multiplication
block? What is the name commonly assigned to this type of computation error, and how would
you correct this? Provide a Verilog example and explain what you would change during the
creation of the corrected peripheral.}

A:

\newpage

\begin{appendices}

\section{Multiplier.c}
\label{appendix:multiplier_c}
\lstinputlisting[style={C-style}]{lab3.c}

\end{appendices}

\end{document}
